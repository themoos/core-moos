

\documentclass[a4paper,10pt]{article}
\usepackage{listings,color,epsfig,amsmath,url}
\definecolor{codecolor}{rgb}{0.99,0.97,0.94} % color values Red, Green, Blue
\definecolor{commentcolor}{rgb}{0.1,0.5,0.1} % color values Red, Green, Blue
\definecolor{stringcolor}{rgb}{0.3,0.1,0.1} % color values Red, Green, Blue
\newcommand{\Code}[1]{\texttt{#1} }
\newcommand{\code}[1]{\Code{#1} }
\newcommand{\DB}   {\code{{MOOSDB}}}
\newcommand{\MA}   {\code{{MOOSApp}}}
\newcommand{\Ignore}[1]   {}



% Title Page
\title{Under the Hood of the MOOS Communications API}
\author{Paul Newman}


\begin{document}
\maketitle

\begin{center}
\epsfig{file=images/moose6.eps,width = 0.2\linewidth}
\end{center}
\begin{abstract}
This document will  give you all the gory details about the mechanisms and protocols
lying at the heart of the MOOS Communications API. The casual or novice user need not worry about many of these details
in the first instance.
\end{abstract}


\section{Introduction}

Much of the text that follows is taken from the original ``MOOS document;;
found on the project website. I have carved this section out as I think it stands alone
as a description of the low level how and why of the MOOS communications architecture.


\subsection{Topology}\label{Sec:Topology}
MOOS has a star-like topology. Each application within a MOOS
community ( a \MA) has a connection to a single ``MOOS Database''
(called \DB) that lies at the heart of the software suite. All
communication happens via this central ``server'' application. The
network has the following properties:
\begin{itemize}
\item No Peer to Peer communication.
\item All communication  between the client and server is instigated by the
client. (ie the \DB {\textbf{never}} makes a unsolicited attempt
to contact a \MA.)
\item Each client has a unique name.
\item A given client need have no knowledge of what other clients exist.
\item A client has no way of transmitting data to a given client -
it can only be sent to  the \DB.
\item The network can be distributed over any number of machines
running any combination of supported operating systems.
\end{itemize}
This centralized topology is obviously vulnerable to
``bottle-necking'' at the server regardless of how well written
the server is. However the advantages of such a design are perhaps
greater than its disadvantages. Firstly the network remains simple
regardless of the number of participating clients. The server has
complete knowledge of all active connections and can take
responsibility for the allocation of communication resources. The
clients operate independently  with inter-connections. This
prevents rogue clients (badly written or hung) from directly
interfering with other clients.

\begin{figure}[ht!]
\centering  \epsfig{figure=./images/Topology.eps,width=0.4\linewidth}
\caption{MOOS binds applications into a network with a star-shaped
topology. Each client has a single communications channel to a
server (\DB).} \label{fig:Topology}
\end{figure}

\section{MOOS Communities (and multiples
of)}\label{sec:Communities}

Post 2002 releases of MOOS have included explicit support
for multiple communities. The idea is that is is sometimes
advantageous to have groups of processes (bound together by mutual
connection to a MOOSDB hub) communicating with each other. To this
end, MOOSMsg's have been extended to included a ``Source
Community'' field which names the community from whence the
message came.

Under the ``one mission one mission file'' paradigm all processes
running  in a community read from the same mission file (see the Programming with MOOS Document and also the Architecture document for more on this paradigm). The community name is specified by
the line \code{Community = <Name> } at the top of the mission
file. \footnote{Perhaps near the \code{ServerPort=} and
\code{ServerHost=} global definitions.}

The question remains --- how does this information get used? Well,
when the MOOSDB serving each community starts up (see Section
\ref{sec:DB}) the community name will be read and all messages
sent from the DB will be tagged with this community name (see
Section \ref{sec:Content}. However this tagging only happens if
the data content originated from the \DB's own local community
--- if it did not (i.e. it came from another community) then the
community name of the orginating community is preserved. To
understand how data from an external community could appear in one
particular \DB see the document on \code{MOOSBridge} for more information.

\section{Message Content}\label{sec:Content}
The communications API in MOOS allows data to be transmitted
between \DB and a client. The meaning of that data is dependent on
the role of the client. However the form of that data is
constrained by MOOS. Somewhat unusually MOOS only allows for data
to be sent in string or double form. Data is packed into messages
(CMOOSMsg class) which contains other salient information shown in
Table \ref{tab:MOOSMsg}.
\begin{table*}
\centering
\begin{tabular}{l|l}\hline
  % after \\: \hline or \cline{col1-col2} \cline{col3-col4} ...
  {\textbf{Variable}} & {\textbf{Meaning}} \\ \hline
  Name  & The name of the data \\
  String Val & Data in string format \\
  Double Val & Numeric double float data \\
  Source & Name of client that sent this data to the \DB \\
  Time & Time at which the data was written \\
  Data Type & Type of data (STRING or DOUBLE)  \\
  Message Type & Type of Message (usually NOTIFICATION) \\
  Source Community & The community to which the source process
belongs---see Sec \ref{sec:Communities}\\\hline
\end{tabular}\vspace{7mm}
  \caption{Contents of MOOS Message}\label{tab:MOOSMsg}
\end{table*}
The fact that data is commonly sent in string format is often seen
as a strange and inefficient aspect of MOOS. For example the
string \verb"Type=EST,Name=AUV,Pos=[3x1]{3.4,6.3,-0.23}" might
describe the position estimate of a vehicle called ``AUV'' as a
3x1 column vector\footnote{Typically string data in MOOS is a
concatenation of comma separated "name = value" pairs.}. It is
true that using custom binary data formats does decrease the
number of bytes sent. However binary data is unreadable to humans
and requires structure declarations to decode it and header file
dependencies are to be avoided where possible. The communications
efficiency argument is not as compelling as one may initially
think. The CPU cost invoked in sending a TCP/IP packet is largely
independent of size up to about one thousand bytes. So it is as
costly to send two bytes as it is one thousand. In this light
there is basically no penalty in using strings. There is however a
additional cost incurred in parsing string data which is far in
excess of that incurred when simply casting binary data.
Irrespective of this, experience has shown that the benefits of
using strings far outweighs the difficulties. In particular:
\begin{itemize}
\item Strings are human readable - debugging is trivial especially
using a tool like MOOSScope. (see the document on Graphical Tools for more on MOOSScope (\code{uMS}) )
\item All data becomes the same type
\item Logging files are human readable (they can be compressed for
storage).
\item Replaying a log file is simply a case of reading strings from
a file and ``throwing'' them back at the \DB in time order.
\item The contents and internal order of strings transmitted by an application can be changed
without the need to recompile consumers (subscribers to that data)
- users simply would not understand new data fields but they would
not crash.
\end{itemize}
Of course, scalar data need not be transmitted in string format -
for example the depth of a sub-sea vehicle. In this case the data
would be sent while setting the data type to  \verb"MOOS_DOUBLE"
and writing the numeric value in the double data field of the
message.

\section{Threading and Process Models}
The choice of processes over threads was made on two counts,
firstly that of stability - a rogue process cannot corrupt the
program/data space of another process (in a sane OS that is).
Secondly on the basis of swift and pain-free development by
several programmers with diverse backgrounds. Building a single
monolithic executable by several people requires, at a minimum,
adherence to programming guidelines and styles that may not be
native to all those included - especially in an academic
environment. The use of small-footprint, independent processes
implies developers can use whatever means they see fit to
accomplish the job. Linking with the communications library
integrates them seamlessly with all other processes but denies a
process the means of interfering with others.



\section{Communications API Mechanics}
%%%%%%%%%%%%%%%% FIGURE %%%%%%%%%%%%%%%%%%%%%%%%%%%%%%
\begin{figure*}[ht!]
\centering \epsfig{figure = ./images/Comms.eps,width = 1.0\linewidth}
\caption{The mechanics of the client server interaction in MOOS.
The user code calls the \code{Notify} method to transmit data.
This method simply adds a message to the ``outbox''. Some time
later (1) the communications thread calls into the database. When
the database is not busy it accepts the client's call (2). The
client then packs the entire outbox into a single large
transmission which is sent to and read by the server (3). The
server unpacks the packet into its constituent messages and place
copies (according to subscriptions and timing ) in the mailboxes
of other connected clients. The server then compresses the mailbox
of the current client into a packet and sends it back to the
client (4). At this point the transaction is then complete and the
server terminates the conversation and looks to begin the same
process with a different client. Upon receiving the reply packet
the client communications thread unpacks it and places the
resulting messages in the ``inbox'' of the client. The user code
can retrieve this list of messages at {\it{any time}} by calling
the \code{Fetch} method.}
\end{figure*}

Each client has a connection to the DB. This connection is made on
the client side by instantiating a class provided in the core
\code{MOOSLIB} library called \code{CMOOSCommClient}. This class
manages a private thread that coordinates the communication with
the \DB. The \code{CMOOSCommClient} object completely hides the
intricacies and timings of the communications from the rest of the
application and provides a small, well defined set of methods to
handle data transfer. Using the \code{CMOOSCommClient} each
application can:
\begin{enumerate}
\item Publish data - issue a notification on named data.
\item Register for notifications on named data.
\item Collect notifications on named  data.
\end{enumerate}

\subsection{Publishing Data - notification} Assume as a result of
 some computation or input, a process \code{A} has a result that
is likely to be of use other undisclosed (remember MOOS
applications do not know about each other) processes \code{B} and
\code{C} - for example a position estimate. The simplest way in
which \code{A} can transmit its new data is to simply call the
\code{Notify} method on its local \code{CMOOSCommClient} object
specifying the data, its name and the time at which it is valid.
Behind the scenes this method then creates a suitable
\code{CMOOSMsg} filling in the relevant fields. The action of
publishing data in this way should be viewed as a
{\emph{notification}} on a named data variable. Crucially this
does not imply a change in the {\emph{value}} - the outcome of the
computations resulting in the need to publish data may remain not
change . For example two sequential estimates of location of a
vehicle may remain numerically the same but the {\it{time}} at
which they are valid changes - this constitutes a data
notification.

As far as the application-specific code in \code{A} is concerned
the invocation of the \code{Notify} method results in the data
being sent to the \DB. What is really happening however is that
the \code{CMOOSCommClient} object is placing the data in an
``outbox''  of \code{CMOOSMsg}s that need to be sent to the \DB at
the next available opportunity. The next obvious question is
``when does the  data reach the \DB?''. The \code{CMOOSCommClient}
object thread has a ``Comms Tick'' - essentially a timer that
contacts the \DB at a configurable rate (typically 10Hz but up to
50Hz ). All the messages in the outbox are packed into a single
packet or ``super message'' called a \code{CMOOSPkt}. Eventually
the \DB will accept the incoming call from the client \code{A} and
receive the packet. At this point the data flow is reversed and
the \DB replies with another \code{CMOOSPkt} containing
notifications (issued by other processes like \code{B} and
\code{C}) that are relevant to  \code{A}. How a process declares
what constitutes a relevant notification is discussed in section
\ref{sec:Registration}. If at the time the thread calls into the
\DB the outbox is empty (i.e there is nothing to notify) a NULL
message is created and sent. Similarly the \DB my reply with a
\code{CMOOSPkt} containing only a NULL message if  nothing of
interest to the client has happened since its last call in. This
policy preserves the strict symmetry of ``one packet sent, one
packet received'' for all occasions.


\subsection{Registration}\label{sec:Registration} Assume that a
list of names of data published has been provided by the author of
a particular MOOS application. For example, a application that
interfaces to a GPS sensor may publish data called \code{GPS\_X}
and \code{GPS\_Y}. A different application may register its
interest in this data by {\it{subscribing }} or {\it{registering}}
for it. An application can register for  notifications using a
single method \code{Register} specifying both the name of the data
and the maximum rate at which the client would like to be informed
that the data has been changed. The latter parameter is specified
in terms of the minimum possible time between notifications for a
named variable. For example setting it to zero would result in the
client receiving in the \code{CMOOSPkt} (that is the \DB's reply)
a collection of messages describing each and every change
notification issued on that variable.

\subsection{Collecting Notifications}\label{sec:CollectingNotifications}
At any time the owner of a \code{CMOOSCommClient} object can
enquire whether it has received any new notifications from the \DB
by invoking the \code{Fetch} method. The function fills in a list
of notification \code{CMOOSMsg}s describing what has changed, what
client made the change, when it was done and what the data type is
(see table \ref{tab:MOOSMsg} in section \ref{sec:Content}). Note
that a single call to \code{Fetch} may result in being presented
with several notifications corresponding to the same named data.
This implies that several changes were made to the data since the
last client-server conversation. However, the time difference
between these similar messages will never be less than that
specified in the \code{Register} function described in section
\ref{sec:Registration}.


\section{The MOOSDB}\label{sec:DB}
The \DB is the heart of the system. It serves as the hub through
which all communication occurs. It is tempting to think of the \DB
as simply a blackboard - an entity which stores the current state
(represented by values of named variables) of the system.
Typically a blackboard allows clients shared access to a
centralized repository of information. Although the \DB does
maintain the most recently set value of all variables and in that
way is similar to a blackboard, the way in which data is retrieved
is very different. The \DB records, on  behalf of its connected
clients, the {\it{history}} of changes to data. Assume a client
\code{A} has subscribed to variables called \code{p} and \code{q}
with a minimum notification period of $\tau_{p}$ and  $\tau_{q}$
seconds respectively. The time is now $t$ and the last call-in was
at time $t'$,$\Delta = t-t'$ seconds ago. When \code{A} calls in,
it is not simply presented with the most recent values of \code{p}
and \code{q} but rather all the {\it{changes}} that have occurred
to them between $t'$ and $t'$. Imagine that some other process
\code{B} is publishing changes (value and/or time-stamp) to $p$
and $q$ every $\tau_B$ seconds. The call in will result in $n_p$
and $n_q$ notification messages being returned to the client where
\begin{align*}
n_p &= \begin{cases}
    floor(\frac{\Delta}{\tau_{p}}) & \text{$\tau_{p} \geq \tau_B$}, \\
    floor(\frac{\Delta}{\tau_{B}}) & \text{otherwise}.
  \end{cases}\\
n_q &= \begin{cases}
    floor(\frac{\Delta}{\tau_{q}}) & \text{$\tau_{q} \geq \tau_B$}, \\
    floor(\frac{\Delta}{\tau_{B}}) & \text{otherwise}.
  \end{cases}
\end{align*}

If no clients have issued relevant notifications since $t'$ then
there will be no notification messages stored at the \DB for
collection by the client. The only exception to this rule is the
first time a client calls in after registering for a notification
on a variable. In this case the value of the variable is returned
in a notification message with a time-stamp specifying when the
data was last set.

\end{document} 