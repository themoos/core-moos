


\documentclass[a4paper,10pt]{article}
\usepackage{listings,color,epsfig,amsmath,url}
\definecolor{codecolor}{rgb}{0.99,0.97,0.94} % color values Red, Green, Blue
\definecolor{commentcolor}{rgb}{0.1,0.5,0.1} % color values Red, Green, Blue
\definecolor{stringcolor}{rgb}{0.3,0.1,0.1} % color values Red, Green, Blue
\newcommand{\Code}[1]{\texttt{#1} }
\newcommand{\code}[1]{\Code{#1} }
\newcommand{\DB}   {\code{{MOOSDB}}}
\newcommand{\MA}   {\code{{MOOSApp}}}
\newcommand{\Ignore}[1]   {}



% Title Page
\title{Scheduling Communications --- \code{pScheduler}}
\author{Paul Newman}


\begin{document}
\maketitle

\begin{center}
\epsfig{file=images/moose6.eps,width = 0.2\linewidth} 
\end{center}
\begin{abstract}
This document will give you a description of \code{pScheduler} ---  a process useful for generating and marhsalling communication packets within a MOOS community.
\end{abstract}

\section{Introduction}\label{Sec:pScheduler}

\code{pScheduler} is a simple tool for generating and responding
to messages send to the \code{MOOSDB} by processes in a MOOS
community. It supports three competencies which will now be
described. Example configuration blocks will also be given.


\begin{description}
\item[SEQUENCES] A looping sequence of messages can be created and published by \code{pScheduler}. Each element of a
sequence is specified in the configuration block  with a line:
\code{SEQUENCE = PUBLISH\_NAME @ VALUE : TIME\_OFFSET }. For
example the configuration in figure \ref{Fig:Sequences} would
write the variable \code{LIGHT\_CONTROL} to the DB every 2 seconds
with its String value alternating between "ON" and "OFF". The
total sequence period is given by the maximum \code{TIME\_OFFSET}
parameter. Figure \ref{Fig:SchedulerExt} shows a more complicated
configuration block in which a sequence is constructed from
several different variables.
\begin{figure}[ht]
\label{Fig:Sequences}
\begin{lstlisting}[]{}
ProcessConfig  = pScheduler
{
    SEQUENCE   = LIGHT_CONTROL @ ON : 2
    SEQUENCE   = LIGHT_CONTROL @ OFF : 4
}
\end{lstlisting}\caption{Configuring a sequence with \code{pScheduler}}
\end{figure}
\item[TIMERS] A ``timer'' allows a variable to be written to the
data base repetitively. A timer can be started and stopped by
publication (by some other application) of user specified
variables. The scheduler can also be told  to derive the  value of
the periodic variable from another MOOS variable, which, if
arrives in the Scheduler's mail box overrides the initial value.
This sounds complicated but isn't. An example is useful. The
general syntax is as follows\\
\code{TIMER = PUBLISH\_NAME @ TIME , START\_VARIABLE,
STOP\_VARIABLE,VALUE\_VARIABLE -> VALUE}\\
Figure \ref{Fig:TIMER} shows a typical configuration block.
\begin{figure}[ht]
\label{Fig:TIMER}
\begin{lstlisting}[]{}
ProcessConfig  = pScheduler 
{
    TIMER = CAMERA_CONTROL @ 4.0 , MISSION_START,  MISSION_END,DESIRED_CAMERA_ANGLE -> 0.0
    TIMER = CAMERA_GRAB @ 4.0, MISSION_START -> GRAB
    TIMER = CAMERA_GRAB @ 4.0,  -> GRAB
}
\end{lstlisting}\caption{Configuring a timer with \code{pScheduler}}
\end{figure}
In this case the variable \code{CAMERA\_CONTROL} will be published
every 4 seconds after some third party application writes
(publishes) the \code{MISSION\_START} variable. If, while the
timer is active, the variable \code{CAMERA\_ANGLE} is published
the value of \code{CAMERA\_CONTROL} will be copied from that
message. If \code{CAMERA\_ANGLE} is never published it will always
have its default value of $0.0$. The timer turns off if any
process publishes \code{MISSION\_END}. The second example is
simpler and writes \code{CAMERA\_GRAB} with value "GRAB" every 4
seconds as soon as \code{MISSION\_START} is written. The third
version starts immediately (note the comma is required...). Figure
\ref{Fig:SchedulerExt} shows a few more examples of the
configuration of timers in  \code{pScheduler}.
\item[RESPONSES] The last competency is one of responding to the
publication of one variable with the publication of one or more
different variables. The syntax is obvious: \code{RESPONSE =
STIMULUS\_VARIABLE : RESPONSE\_VARIABLE @
VALUE,RESPONSE\_VARIABLE2 @ VALUE,....  }. Here
\code{STIMULUS\_VARIABLE} is the name of the variable we wish
\code{pScheduler} to respond to and after the colon comes a
command separated list of response variables with the values they
should contain. Figure \ref{Fig:SchedulerExt} shows some clear
example configurations of "responses" in \code{pScheduler}
\end{description}

\begin{figure}
\label{Fig:SchedulerExt}
\begin{lstlisting}[]{}
ProcessConfig = pScheduler
{
    //generate a sequence 6 seconds long ....
    // VAR1 will fire after 1 second
    // VAR2 fire after 3 seconds
    // VAR3 fire after 6 seconds
    // one second later VAR1 will fire again..repeat...
    SEQUENCE = VAR1 : RED @ 1 SEQUENCE = VAR2 : ORANGE @ 3 SEQUENCE =
    VAR3 : GREEN @ 6

    //generate a timer that writes "VAR_T1" with value "TimerData"
    TIMER = VAR_T1 @ 3.0, -> TimerData

    // generate a timer that writes "VAR_T2" with a string version
    // of the current value of DB_TIME (published by the DB) TIMER =
    VAR_T2 @ 2.0 ,,,DB_TIME -> TimerData

    // generate a timer that writes "VAR_T3" with the current
    // value of DB_TIME (published by the DB)
    // which only starts when "GO_T3" is published and stops
    // when "STOP_T3" is published
    TIMER = VAR_T3 @ 4.0,GO_T3,STOP_T3,DB_TIME -> TimerData

    //generate a response to "SURPRISE_ME". The variable "BOO"
    // takes on string value "HOO" and variable R9 has value
    //"get_a_grip"
    RESPONSE = SURPRISE_ME : BOO @ HOO, R9 @ get_a_grip

    //generate a response to "DB_TIME".
    RESPONSE = DB_TIME : ACKNOWLEDGMENT @ I_GOT_THE_TIME

}
\end{lstlisting}\caption{An extended \code{pScheduler} configuration block}
\end{figure}
\end{document}
