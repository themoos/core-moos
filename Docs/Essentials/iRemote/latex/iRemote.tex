


\documentclass[a4paper,10pt]{article}
\usepackage{listings,color,epsfig,amsmath,url,tabularx}
\definecolor{codecolor}{rgb}{0.99,0.97,0.94} % color values Red, Green, Blue
\definecolor{commentcolor}{rgb}{0.1,0.5,0.1} % color values Red, Green, Blue
\definecolor{stringcolor}{rgb}{0.3,0.1,0.1} % color values Red, Green, Blue
\newcommand{\Code}[1]{\texttt{#1} }
\newcommand{\code}[1]{\Code{#1} }
\newcommand{\DB}   {\code{{MOOSDB}}}
\newcommand{\MA}   {\code{{MOOSApp}}}
\newcommand{\Ignore}[1]   {}



% Title Page
\title{The Human Touch with \code{iRemote}}
\author{Paul Newman}


\begin{document}
\maketitle

\begin{center}
\epsfig{file=images/moose6.eps,width = 0.2\linewidth} 
\end{center}
\begin{abstract}
This document will give you a description of how to use and configure \code{iRemote} to inject commands into a community and 
also to query the status of certain commonly used variables from a terminal.
\end{abstract}

\section{Manual Control - \code{iRemote}}\label{sec:iRemote}
%
\code{iRemote} was designed to be a control terminal for a
deployed vehicle. It is really nothing more than a long
\code{switch} statement based on characters input from the
keyboard. One of its many functions is to allow remote control of
the actuators of the vehicle. This is an invaluable asset for land
and sub-sea vehicles alike. The application is multithreaded. The
primary thread blocks on a read of keyboard input. When a
character is pressed some action is taken - for example publishing
a new value for \code{DESIRED\_THRUST}. The fact that \code{iRemote}
can take control of a real vehicle presents a safety problem. What
if the human controller walks away or even worse the vehicle moves
out of communication range (eg a submarine dives) and the console
is not available? To prevent the last issued actuator command
being carried out {\it{ad-infinitum }} a secondary in thread
\code{iRemote} prompts the user to hit an acknowledge key (') at
least every 15 seconds. If the human driver does not respond the
all actuators are set to the zero position.\footnote{The actuation
driver  in iActuation will also shut down all motors if it does
not receive control commands for an extended period of time. In
the Bluefin vehicle the driver class within iActuation sends a
message to the Janitor processes which resets a watchdog on the
power management board - keeping the tail cone powered.}

\subsection{Summary of Functionality}
The following (not exhaustive) list describes some of the online
functionality that \code{iRemote} provides:

\begin{tabularx}{\linewidth}{l|c|X}
\textbf{Function} & \textbf{Key} & \textbf{Comment} \\ \hline

 Reload Mission File & \textsc{R} & Tells \code{pHelm} to
rebuild its
tasks \\
%
Restart Navigation & \textsc{V} & Tells \code{pNav} to reboot all
navigation filters
\\
%
Restart Logger  & \textsc{G} & Tells \code{pLogger} to begin
recording to a new set of log files \\
%
Begin Mission & \textsc{O} & Instructs \code{pHelm} to go online\\
%
Halt Mission & \textsc{O} or {\it{space}} & \code{pHelm} goes
offline and \code{iRemote} takes control immediately. \\
%
Navigation Summary & \textsc{*} & Prints a summary of salient
navigation information \\
%
Rudder Left/Right & \textsc{n},\textsc{m} & Steer control \\
%
Elevator Up/Down & \textsc{p},\textsc{l} & Pitch control \\
%
Thrust Up/Down & \textsc{a},\textsc{z} & Throttle control (+ shift
gives 100 percent) \\
%
Stop        & {\it{space}} & Immediate zero of all degrees of
freedom \\
%
Fetch DB       & \textsc{F} & Prints a summary of the contents of the entire \DB \\
%
CustomKey & $[0 \rightarrow 9]$ & The numeric keys can be made
(via \code{iRemote}s configuration block) to publish any named
variable with a specified value. In the example configuration
block below (Figure \ref{fig:iRemoteBlock}), pressing  key ``2''
will cause \code{iRemote} to write the variable
\code{JANITOR\_SWITCH} with the string value (quotes)
ACTUATION::OFF \\
%
CustomSummary & \textsc{+} & The configuration block allows a
custom summary to be built consisting of any variable names used
within the system. \code{iRemote} subscribes to this data and
prints its current value when requested.\\
%
{{CustomJournal}} & $[0 \rightarrow 9]$ & Similar to
{\it{CustomSummary}} but instead of keeping the most recently
published variable it keeps a history of values. Each Journal can
be  bound to a numeric key. In the example below pressing key
``6'' will show the past 10 values of \code{DESIRED\_RUDDER} with
every delta captured as the
capture time is 0.\\
\end{tabularx}

\begin{figure}[ht]
\begin{lstlisting}[ ]{}
///////////////////////////////////////////////
// iRemote config block
ProcessConfig = iRemote
{
    CustomJournal = Name = DESIRED_RUDDER,Key =6, History = 10,Period = 0
    CustomSummary = DESIRED_THRUST
    CustomKey = 2 : JANITOR_SWITCH @ "ACTUATION:OFF"
}
\end{lstlisting}
\caption{A small configuration block for iRemote showing typical
usage of the \code{CustomX} commands}\label{fig:iRemoteBlock}
\end{figure}


\subsection{Informing the Pilot}
In most missions \code{iRemote} is the only interface the vehicle
pilot has with the vehicle. Clearly then a method is needed by
which {\em{important}} information can be sent to the
\code{iRemote} console from any process. The \code{CMOOSApp}
member function \code{MOOSDebugWrite} achieves this by issuing a
notification on a variable watched by \code{iRemote}. Such
messages are displayed on the iRemote console at run time along
with the process making the announcement. Note that this  name is
somewhat unfortunate as this function should not be used for
debugging - it is a run-time thing. It is frustrating to have a
cornucopia of messages flashing on the screen during a mission the
content of which is meaningless to the pilot. Typical uses of this
functionality would be a very occasional summary of navigation
status and system level warning messages - for example
notification of unexpected mission task  termination.
\end{document}
%